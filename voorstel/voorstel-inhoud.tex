%---------- Inleiding ---------------------------------------------------------

% TODO: Is dit voorstel gebaseerd op een paper van Research Methods die je
% vorig jaar hebt ingediend? Heb je daarbij eventueel samengewerkt met een
% andere student?
% Zo ja, haal dan de tekst hieronder uit commentaar en pas aan.

%\paragraph{Opmerking}

% Dit voorstel is gebaseerd op het onderzoeksvoorstel dat werd geschreven in het
% kader van het vak Research Methods dat ik (vorig/dit) academiejaar heb
% uitgewerkt (met medesturent VOORNAAM NAAM als mede-auteur).
% 

\section{Inleiding}
\label{sec:inleiding}
Deze bachelorproef richt zich op de toepassing van machine learning voor de voorspellen van smaakprofiel en monitoring van fermentatieproces in traditionele lambiekbrouwerijen. Het thema combineert  traditionele ambachtelijke productie met moderne data-analysetechnieken en richt zich specifiek op \textbf{brouwers die lambiek produceren} in het Pajottenland en de Zennevalei.
De centrale probleemstelling is dat spontane gisting leidt tot een hoge variabiliteit tussen batchen en vaten, waardoor kwaliteitsconsistentie en voorspelbaarheid van het productieproces beperkt zijn. Hoewel veel brouwerijen proces- en sensorische data verzamelen wordt die zelden benut om voorspelling mee uit te voeren.

\paragraph{Centrale onderzoeksvraag}
{\color{purple}Hoe kan een machine-learningmodel ontwikkeld worden dat fermentatiegedrag en kwaliteitsindicatoren van lambiek voorspelt, zodat traditionele brouwerijen hun productieproces beter kunnen sturen en de kwaliteit van het eindproduct kunnen waarborgen?}

\paragraph{Onderzoeksdoelstelling}
Het doel van dit onderzoek is het ontwikkelen van een reproduceerbaar Python-model dat fermentatie en kwaliteitsparameters kan voorspellen op basis van historische en real-time data. Wanneer deze modellen nauwkeurig en stabiel blijken (\textbf{Scenario A}) wordt het model aangevuld met een prototype-dashboard dat datagedreven monitoring voor brouwers ondersteunt. Indien de voorspellingen onvoldoende betrouwbaar zijn (\textbf{Scenario B}) verschuift de focus op het onderzoek naar alternatieve AI-technieken (zoals neurale netwerken of unsupervised learning) om de onderliggende variabiliteit en patronen in spontane gisting beter te begrijpen. In beide scenario\textquotesingle s is het doel een proof-of-concept op te leveren dat bruikbare inzichten biedt voor traditionele lambiekbrouwerijen.

\paragraph{Deelvragen probleemdomein}
{\color{purple}
\begin{itemize}
  \item Welke factoren (microbiële populaties, vatvariatie, temperatuur, luchtkwaliteit) beïnvloeden de variabiliteit in spontane lambiekfermentatie het meeste?
  \item Welke fermentatieparameters (pH, suikerafbraak, alcoholpercentage, aromaprofiel) zijn essentieel voor het voorspellen van kwaliteit en smaak?
  \item Welke datakwaliteitsproblemen (ontbrekende waarden, inconsistenties tussen batches of vaten) kunnen de prestaties van ML-modellen beïnvloeden?
  \item Welke bestaande kennis of modellen zijn er over fermentatei en lambiekproductie en wat zijn hun beperkingen voor het voorspellende doeleinden?
\end{itemize}}

\paragraph{Deelvragen oplossingsdomein}
{\color{purple}
\begin{itemize}
  \item Welke machine-learningmodellen (bijv. lineaire regressie, Random Forest, XGBoost) presteren het best bij het voorspellen van fermentatieparameters en smaakprofielen?
  \item Welke feature-engineeringsstrategieën (tijdvensters, aggregatie, trenddetectie) verhogen de voorspellende waarde van het model?
  \item Hoe kan het model gevaldeerd worden op onafhankelijke batches of tijdreeksen om reproduceerbaarheid en generaliserbaarheid te waarborgen?
  \item Welke alternatieve AI-technieken (unsupervised learning, anomaly detection) kunnen ingezet worden wanneer klassieke ML-modellen onvoldoende presteren?
\end{itemize}}

% \begin{itemize}
%   \item kaderen thema
%   \item de doelgroep
%   \item de probleemstelling en (centrale) onderzoeksvraag
%   \item de onderzoeksdoelstelling
% \end{itemize}

% Denk er aan: een typische bachelorproef is \textit{toegepast onderzoek}, wat betekent dat je start vanuit een concrete probleemsituatie in bedrijfscontext, een \textbf{casus}. Het is belangrijk om je onderwerp goed af te bakenen: je gaat voor die \textit{ene specifieke probleemsituatie} op zoek naar een goede oplossing, op basis van de huidige kennis in het vakgebied.

% De doelgroep moet ook concreet en duidelijk zijn, dus geen algemene of vaag gedefinieerde groepen zoals \emph{bedrijven}, \emph{developers}, \emph{Vlamingen}, enz. Je richt je in elk geval op it-professionals, een bachelorproef is geen populariserende tekst. Eén specifiek bedrijf (die te maken hebben met een concrete probleemsituatie) is dus beter dan \emph{bedrijven} in het algemeen.

% Formuleer duidelijk de onderzoeksvraag! De begeleiders lezen nog steeds te veel voorstellen waarin we geen onderzoeksvraag terugvinden.

% Schrijf ook iets over de doelstelling. Wat zie je als het concrete eindresultaat van je onderzoek, naast de uitgeschreven scriptie? Is het een proof-of-concept, een rapport met aanbevelingen, \ldots Met welk eindresultaat kan je je bachelorproef als een succes beschouwen?

%---------- Stand van zaken ---------------------------------------------------

\section{Literatuurstudie}%
\label{sec:literatuurstudie}
Het brouwen van lambiekbieren onderscheidt zich door een uniek spontaan fermentatieproces, waarbij het wort niet wordt geïnoculeerd met een specifieke gistcultuur, maar wordt blootgesteld aan micro-organismen uit de lucht en de houten vaten \autocite{BoonSpontaneous2025,EylenboschSpontaneous2025}. Deze spontane gisting zorgt voor een rijke microbiële diversiteit die de smaak en de zuurtegraad van het bier sterk beïnvloeden. De aanwezigheid van azijnzuurbacteriën en andere micro-organismen in de vaten is niet uniform, maar varieert zowel in tijd als in ruimte, wat bijdraagt aan de dynamische aard van het fermentatieproces en de unieke kenmerken van elke batch \autocite{DeRoosVerceAertsVandammeDeVuyst2018}.

Door deze natuurlijke variabiliteit is het voorspellen van het exacte fermentatieverloop complex. Traditionele wiskundige modellen bieden een basis, maar zij kunnen beperkt omgaan met niet-lineaire interacties tussen micro-organismen, omgevingsfactoren en vatvariëteit. Black-boxmodel zoals neurale netwerken  blijken effectiever in het herkennen van patronen in historische data en het voorspellen van fermentatieparameters zoals suikerafbraak en zuurtegraad \autocite{TreleaTiticaLandaudLatrilleCorrieu2001}.

De recente opkomst van machine learning in het brouwproces biedt extra mogelijkheden aan om deze complexiteit te beheersen. Door historische en real-time data te combineren, kunnen algoritmes zoals random forrest, gradiant boosting en deep learning trends kwaliteitsafwijkingen detecteren en procesoptimalisatie voorstellen \autocite{Nettesheim2024}. Deze benadering maakt het mogelijk om vroegtijdig inzicht te krijgen in de mogelijke aroma\textquotesingle s in de lambiek.

Samengevat benadrukt de literatuur dat een combinatie van diepgaande microbiologische inzichten en moderne machine-learningmodellen essentieel is om spontane gisting in lambiekbrouwerijen te begrijpen en te monitoren. Deze geïntegreerde aanpak stelt brouwers in staat om kwaliteitsconsistentie te verhogen en vroegtijdig inzicht te krijgen in de aroma\textquotesingle s terwijl de ambachtelijke eigenschappen van lambiekbieren te behouden \autocite{Bokulich2014,DeRoosVerceAertsVandammeDeVuyst2018,TreleaTiticaLandaudLatrilleCorrieu2001,Nettesheim2024,BoonSpontaneous2025,EylenboschSpontaneous2025}



% Hier beschrijf je de \emph{state-of-the-art} rondom je gekozen onderzoeksdomein, d.w.z.\ een inleidende, doorlopende tekst over het onderzoeksdomein van je bachelorproef. Je steunt daarbij heel sterk op de professionele \emph{vakliteratuur}, en niet zozeer op populariserende teksten voor een breed publiek. Wat is de huidige stand van zaken in dit domein, en wat zijn nog eventuele open vragen (die misschien de aanleiding waren tot je onderzoeksvraag!)?

% Je mag de titel van deze sectie ook aanpassen (literatuurstudie, stand van zaken, enz.). Zijn er al gelijkaardige onderzoeken gevoerd? Wat concluderen ze? Wat is het verschil met jouw onderzoek?

% Verwijs bij elke introductie van een term of bewering over het domein naar de vakliteratuur, bijvoorbeeld~\autocite{Hykes2013}! Denk zeker goed na welke werken je refereert en waarom.

% Draag zorg voor correcte literatuurverwijzingen! Een bronvermelding hoort thuis \emph{binnen} de zin waar je je op die bron baseert, dus niet er buiten! Maak meteen een verwijzing als je gebruik maakt van een bron. Doe dit dus \emph{niet} aan het einde van een lange paragraaf. Baseer nooit teveel aansluitende tekst op eenzelfde bron.

% Als je informatie over bronnen verzamelt in JabRef, zorg er dan voor dat alle nodige info aanwezig is om de bron terug te vinden (zoals uitvoerig besproken in de lessen Research Methods).

% Voor literatuurverwijzingen zijn er twee belangrijke commando's:
% \autocite{KEY} => (Auteur, jaartal) Gebruik dit als de naam van de auteur
%   geen onderdeel is van de zin.
% \textcite{KEY} => Auteur (jaartal)  Gebruik dit als de auteursnaam wel een
%   functie heeft in de zin (bv. ``Uit onderzoek door Doll & Hill (1954) bleek
%   ...'')

% Je mag deze sectie nog verder onderverdelen in subsecties als dit de structuur van de tekst kan verduidelijken.

%---------- Methodologie ------------------------------------------------------
\section{Methodologie}%
\label{sec:methodologie}
De methodologie van dit onderzoek volgt de experimentele aanpak. Het doel is om te onderzoeken in welke mate machine-learningmodellen in staat zijn om patronen te vinden in spontane lambiekfermentatieprocessen en of deze modelen een bruikbaar inzicht kunnen geven voor kwaliteitsoptimalizatie.

\subsection*{Onderzoeksopzet}
Het onderzoek wordt uitgevoerd in vier hoofdfasen: literatuurstudie, dataverwerking, modellering en evaluatie, gevolgd door een beslissingsfase waarin bepaald wordt of de proof-of-concept-dashboard ontwikkeld wordt of dat alternatieve AI-methoden onderzocht worden.

\subsection*{Dataverzameling en -voorbereiding}
De beschikbare datasets uit traditionele lambiekbrouwerijen worden verzameld, opgeschoond en voorbereid voor analyse. Dit omvat het behandelen van ontbrekende waarden, het verwijderen van ruis en feature engineering om relevante variabelen zoals temperatuur, pH , dichtheid en microbiële indicatoren geschikt te maken voor machine learning. De dataset wordt opgesplitst in training-, evaluatie- en testsets volgens gangbare best practices.

\subsection*{Modellering}
Verschillende machine learningmodellen worden ontwikkeld, waaronder:
\begin{itemize}
  \item  Basisline-modellen zoals lineaire regressie
  \item  Ensemble-modellen zoals Random Forrest en XGBoost
\end{itemize}

Hyperparameters worden geoptimaliseerd en de prestaties van elk model worden geëvalueerd aan de hand van metrics zoals RMSE en MAE.

\subsection*{Evaluatie en beslissingsmoment}
De modellen worden vergeleken op nauwkeurigheid, stabiliteit en reproduceerbaarheid. Afhankelijk van de resultaten wordt volgende beslissing gemaakt.
\begin{itemize}
  \item \textbf{Scenario A: modellen zijn voldoende vertrouwbaar.} In dit geval wordt een PoC-dashboard ontwikkeld om voorspellingen en trends toegankelijk te maken voor de brouwer.
  \item  \textbf{Scenario B: modellen zijn niet voldoende vertrouwbaar.} Indien klassieke machine-learningmodellen niet in staat blijken om fermentatieparameters met voldoende nauwkeurigheid te voorspellen wordt geen dashboard ontwikkeld. In plaats daarvan verschuift de focus naar het onderzoeken van alternatieve AI-technieken die beter geschikt zijn voor complexe niet-lineaire en variabele tijdsreeksen.
          In dit scenario worden onder meer de volgende pistes onderzocht:
          \begin{itemize}
          \item \textbf{Neurale netwerken} voor het modelleren van niet-lineaire tijdsreeksen.
          \item \textbf{Unsupervised learning} om verborgen patronen, variatie of microbiële fases te identificeren zonder labels.
          \item \textbf{Anomaly detection-modellen} om fermentatieafwijkingen vroegtijdig te detecteren.
          \end{itemize}
          Het eindresultaat van Scenario B is een vergelijkend onderzoek waarin duidelijk wordt waarom klassieke ML-methoden tekortschieten en welke AI-technieken potentieel beter geschikt zijn voor spontane lambiekfermentatie.
        
\end{itemize}

\subsection*{Tools en technologieën}
Voor dit onderzoek worden onder andere de volgende tools gebruikt:
\begin{itemize}
    \item Python (NumPy, Pandas, Scikit-learn, Matplotlib, eventueel TensorFlow of PyTorch)
    \item Jupyter Notebooks voor experimenten en documentatie
    \item Git voor versiebeheer
\end{itemize}

\subsection*{Tijdsplanning en deliverables (13 weken)}
\begin{itemize}
  \item \textbf{Week 1--2:} Literatuurstudie en afbakening van het probleem. \emph{Deliverable: literatuurhoofdstuk en onderzoeksvragen.}
  \item \textbf{Week 3:} Verzameling en eerste verkenning van de dataset. \emph{Deliverable: eerste EDA-notebook.}
  \item \textbf{Week 4--5:} Datapreprocessing en feature engineering. \emph{Deliverable: opgeschoonde dataset en preprocessing-scripts.}
  \item \textbf{Week 6--7:} Ontwikkeling van baseline en klassieke ML-modellen. \emph{Deliverable: modelresultaten.}
  \item \textbf{Week 8:} Hyperparameter-tuning en testen van geavanceerde modellen. \emph{Deliverable: geoptimaliseerde modellen.}
  \item \textbf{Week 9:} Evaluatie van modellen en beslissingsmoment. \emph{Deliverable: evaluatierapport.}
  \item \textbf{Week 10--12:} Scenario-afhankelijke actie: dashboardontwikkeling of AI-vervolgonderzoek. \emph{Deliverable: PoC-dashboard of AI-experimentrapport.}
  \item \textbf{Week 13:} Eindredactie en afronding van de bachelorproef. \emph{Deliverable: finale tekst en bijlagen.}
\end{itemize}

Deze methodologie zorgt ervoor dat het onderzoek zowel technisch diepgaand is als objectief meetbaar is en dat de resultaten reproduceerbaar en wetenschappelijk verantwoord is.


% Hier beschrijf je hoe je van plan bent het onderzoek te voeren. Welke onderzoekstechniek ga je toepassen om elk van je onderzoeksvragen te beantwoorden? Gebruik je hiervoor literatuurstudie, interviews met belanghebbenden (bv.~voor requirements-analyse), experimenten, simulaties, vergelijkende studie, risico-analyse, PoC, \ldots?

% Valt je onderwerp onder één van de typische soorten bachelorproeven die besproken zijn in de lessen Research Methods (bv.\ vergelijkende studie of risico-analyse)? Zorg er dan ook voor dat we duidelijk de verschillende stappen terug vinden die we verwachten in dit soort onderzoek!

% Vermijd onderzoekstechnieken die geen objectieve, meetbare resultaten kunnen opleveren. Enquêtes, bijvoorbeeld, zijn voor een bachelorproef informatica meestal \textbf{niet geschikt}. De antwoorden zijn eerder meningen dan feiten en in de praktijk blijkt het ook bijzonder moeilijk om voldoende respondenten te vinden. Studenten die een enquête willen voeren, hebben meestal ook geen goede definitie van de populatie, waardoor ook niet kan aangetoond worden dat eventuele resultaten representatief zijn.

% Uit dit onderdeel moet duidelijk naar voor komen dat je bachelorproef ook technisch voldoen\-de diepgang zal bevatten. Het zou niet kloppen als een bachelorproef informatica ook door bv.\ een student marketing zou kunnen uitgevoerd worden.

% Je beschrijft ook al welke tools (hardware, software, diensten, \ldots) je denkt hiervoor te gebruiken of te ontwikkelen.

% Probeer ook een tijdschatting te maken. Hoe lang zal je met elke fase van je onderzoek bezig zijn en wat zijn de concrete \emph{deliverables} in elke fase?

%---------- Verwachte resultaten ----------------------------------------------
\section{Verwacht resultaat, conclusie}%
\label{sec:verwachte_resultaten}
Het verwachte resultaat is een proof-of-concept machinelearningmodellen dat fermentieparameters van lambiekbieren kan voorspellen op basis van historische en real-time data.
Indien de voorspelling betrouwbaar zijn (\textbf{Scenario A}) wordt er een eenvoudig dashboard ontwikkeld waarin trends en voorspellingen van bijvoorbeeld pH , dichtheid en alcoholpercentage per batch wordt weergegeven. Als de modellen onvoldoende betrouwbaar zijn (\textbf{Scenario B}) wordt onderzocht of geavanceerde AI-technieken zoals neurale netwerken of unsupervised learning verborgen patronen kunnen detecteren.

De meerwaarde voor traditionele lambiekbrouwerijen ligt in het vroegtijdig opsporen van afwijkingen het beter monitoren van fermentatieproces en de smaak aroma\textquotesingle s kunnen voorspellen zonder te proeven zonder de ambachtelijke karakters aan te passen.

De verwachte conclusies zijn dat machine learning modellen een deel van de variabiliteit in spontane gisting kunnen voorspellen en dat klassieke modelen mogelijk onvoldoende zijn voor een dashboard, terwijl complexere AI-benaderingen aanvullende inzichten kunnen bieden bij grotere variabiliteit.
% Hier beschrijf je welke resultaten je verwacht. Als je metingen en simulaties uitvoert, kan je hier al mock-ups maken van de grafieken samen met de verwachte conclusies. Benoem zeker al je assen en de onderdelen van de grafiek die je gaat gebruiken. Dit zorgt ervoor dat je concreet weet welk soort data je moet verzamelen en hoe je die moet meten.

% Wat heeft de doelgroep van je onderzoek aan het resultaat? Op welke manier zorgt jouw bachelorproef voor een meerwaarde?

% Hier beschrijf je wat je verwacht uit je onderzoek, met de motivatie waarom. Het is \textbf{niet} erg indien uit je onderzoek andere resultaten en conclusies vloeien dan dat je hier beschrijft: het is dan juist interessant om te onderzoeken waarom jouw hypothesen niet overeenkomen met de resultaten.

